\section{Experimental Evaluation}
\label{sec:experiments}

% This section should, together with Section~\ref{sec:implementation},
% constitute roughly half of the paper (implementation + evaluation).

\subsection{Experimental Setup}
\label{sec:exp_setup}
% Describe the evaluation protocol:
% - Which datasets and tasks you evaluated on
% - Train/validation/test splits
% - Metrics used (e.g., mAP, IoU, ADE/FDE, accuracy)
% - Baselines (if any) and how you compare against them

\subsection{Quantitative Results}
\label{sec:exp_quant}
% Present tables of your main quantitative results.
% Compare to numbers reported in the original paper when possible,
% and clearly mark results that are not directly comparable.

\subsection{Qualitative Analysis}
\label{sec:exp_qual}
% Show visualizations (qualitative examples, heatmaps, BEV maps, etc.).
% Highlight both typical successes and informative failure cases.

\subsection{Ablation Studies}
\label{sec:exp_ablation}
% If feasible, include small ablations:
% - Effect of key architectural components
% - Effect of important hyperparameters
% - Trade-offs between accuracy and efficiency
