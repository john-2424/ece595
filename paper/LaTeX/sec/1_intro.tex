
\section{Introduction}
\label{sec:intro}

Research in computer vision continues to evolve rapidly, driven by advances in network architectures, multimodal perception, and learning paradigms for autonomous and interactive systems. For academic and practical understanding, reviewing recent progress in top‑tier venues such as CVPR provides valuable insights into both conceptual innovations and empirical methodologies. As part of course project work, this paper examines three recent CVPR papers—each representative of a different subdomain within vision—through detailed critique and structured analysis.

The first chosen work, ConvNeXt (CVPR 2022)~\cite{liu2022convnext}, modernizes convolutional architectures by carefully re‑designing components inspired by Vision Transformers while retaining the efficiency and locality benefits of CNNs. The second work, UniAD (CVPR 2023)~\cite{li2023uniad}, proposes an end‑to‑end framework for autonomous driving that unifies perception, prediction, and planning into a single coordinated model. The third work, a 2025 CVPR paper on reasoning‑based visual navigation~\cite{nav2025dynamics}, reframes agent policy learning through a dynamical‑systems interpretation, enabling controllable and interpretable navigation behavior.

Although these three papers differ in application domain, scale, and modeling philosophy, they collectively illustrate the diversity and maturity of modern computer vision research. The goal of this project is twofold: (i) to provide a substantive critique of the selected papers, and (ii) to outline the foundations for implementing one of them as a future course deliverable. The critiques focus on modeling assumptions, empirical rigor, reproducibility, and limitations, while the implementation section establishes the structure required for a non‑trivial reproduction effort.

The remainder of this paper is organized as follows. Section~\ref{sec:background} provides general background on the problem settings underlying the three works. Section~\ref{sec:critiques} presents a detailed critique of each paper. Section~\ref{sec:implementation} outlines the components required for a faithful implementation of a chosen method. Section~\ref{sec:experiments} describes an experimental framework to be completed in future work. Finally, Sections~\ref{sec:discussion} and \ref{sec:conclusion} offer broader reflections and concluding remarks.
