\section{Conclusion}
\label{sec:conclusion}

% Briefly summarize:
% - The set of four papers you reviewed and critiqued
% - The method you implemented and evaluated
% - The key empirical and conceptual takeaways from your study
%
% Keep this section concise (typically one paragraph).


In this report, I have begun assembling a unified view of recent advances in both backbone architectures and task-specific 3D perception models. 
The first paper, ConvNeXt, revisits convolutional networks in the era of Vision Transformers and demonstrates that with carefully chosen design updates and training protocols, 
purely convolutional backbones can match or surpass transformer-based models on standard 2D recognition benchmarks. 
The second paper, BEV-SAN, shifts the focus from generic backbones to the geometry of bird's-eye-view representations for camera-only 3D detection, 
showing that explicitly modeling height structure through slice-based attention can substantially improve performance, particularly for small and low-lying objects. 
Together, these works illustrate how architecture-level and representation-level choices jointly determine the effectiveness of modern computer vision systems and set the stage for further analysis of additional papers and an eventual implementation and evaluation of one selected method.

